En conclusion, ce projet a permis aux étudiants concernés de s'immiscer dans l'ensemble des domaines de l'informatique, de l'électronique et de l'électromécanique dans le but de réaliser un produit permettant la coupe automatique de fil électrique. En effet, celui-ci est passé par diverses étapes telles que la modélisation et l'impression de pièces en 3D, la programmation d'un microcontrôleur, et l'assemblage manuel d'un prototype. Le résultat final permet la coupe de fil électrique de manière automatisée, simple et précise. Malgré le fait que des améliorations soient bien entendu possibles, telles que le choix du dénudage du fil, l'objectif principal du projet a pu être atteint.