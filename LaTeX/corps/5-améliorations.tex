\par Enfin, même si le prototype final obtenu permet de réaliser la tâche élémentaire de couper du fil automatiquement, celui-ci est loin d'être parfait, et pourrait bien entendu recevoir de nombreuses améliorations.
\par La plus évidente serait de pouvoir permettre la fonction de dénudage, comme cela avait été prévu au début du projet. À l'instant actuel, cette fonction n'a pas été activée dans le code du microcontrôleur, car elle ne complète sa tâche que rarement, étant donné la position très précise que le fil doit avoir à la sortie du tube issu du moteur à tirer. Une amélioration dans ce cas-ci serait éventuellement de choisir un tube plus fin par lequel passerait le fil et de rapprocher à l'extrême la pince à dénuder de sa sortie pour essayer d'assurer une précision maximale, et permettre la fonction de dénudage.
\par Si ceci est réalisable, il est alors aussi plausible de penser à une adaptation des épaisseurs de fils à dénuder. En effet, la pince à dénuder multi-fonctions choisie possède des indentations de nombreuses tailles pour différents types de fils. S'il est possible d'atteindre une précision permettant le dénudage d'une certaine épaisseur de fil, alors il devrait pouvoir en découler l'option de choisir ce paramètre avant de lancer l'exécution du prototype.
\par Enfin, une idée qui avait été discutée au début du projet fut celle de l'ajout d'un composant avec une interface graphique, permettant une interaction en temps réel avec l'utilisateur lors du fonctionnement du prototype. Ce composant pourrait par exemple prendre la forme d'un petit écran tactile où s'afficheraient les paramètres tels que la longueur du fil à couper dans l'état actuel du prototype, mais aussi la longueur à dénuder et l'épaisseur du fil traité dans le cas des améliorations discutées plus haut.