Pour utiliser le prototype, il faut en premier lieu brancher celui-ci à deux sources d'alimentation :
\begin{itemize}
    \item sur le breadboard dans les colonnes dédiées à l'alimentation de 5V et la masse ;
    \item sur le microcontrôleur via une des entrées USB présentes (soit le Micro-USB, soit l'USB-A).
\end{itemize}

\par Lorsque le prototype est alimenté, celui-ci démarre automatiquement, et tirera et coupera du fil électrique fourni par la bobine de manière continue jusqu'à ce que l'alimentation du prototype soit interrompue manuellement.

\par Éventuellement, il est aussi possible de modifier le code du microcontroleur en utilisant PSoC Creator ; notamment en modifiant l'argument de \verb|turn_feeder_motor()| dans le \verb|main()|, étant donné que celui-ci indique la longueur en mm du fil qui sera tiré. Ainsi, en modifiant ce paramètre, il est possible d'avoir des longueurs de fil coupé de longueur arbitraire.